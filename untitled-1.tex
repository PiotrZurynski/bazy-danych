\documentclass[a4paper,12pkt]{amsart} 
\usepackage[T1]{fontenc}
\title{WPROWADZENIE DO OPERATORÓW RÓŻNICZKOWYCH NA GRAFACH} 
\author{Kamil Reszko} 

\begin{document} 
\maketitle 
\section*{STRESZCZENIE.} Krótkie wprowadzenie do operatorów różniczkowych na grafach: laplasjan oraz operator prawdopodobieństwa 
\section*{NOTIONS} 
$\mathbb{N}$ - zbiór liczb naturalnych \newline 
\indent $\mathbb{R}$ - zbiór liczb rzeczywisty \newline                                            
\indent $\mathbb{R}^{+}$ - zbiór liczb rzeczywistych dodatnich \newline 
\indent \# - ilość elementów w zbiorze. 
\section{GRAFY WAŻONE} 
\textit{Graf ważony (inaczej graf z wagami)} jest parą \textit{(V,b)}, gdzie V jest zbiorem wierzchołków (tj. dowolnym zbiorem) oraz  b: V $\times$ V $\rightarrow\mathbb{R}$ spełnia następujące warunki:
\indent $\textit{b(x,y)} \leq 0$ any $\textit{x,y} \in V$, \newline 
\indent $\textit{b(x,y)=b(y,x)}$ for any $\textit{x,y} \in V$, \newline 
\indent ${b(x,x) = 0}$ for any \textit{x} $\in$ V. \newline 
\indent Jeśli ${(x,y)} \ne 0$, to mówimy, że pomiędzy \textit{x,y} jest \textit{krawędź} i zapisujemy\textit{x} $\sim$ \textit{y}. Trzeci warunek oznacza, że rozważamy grafy bez pętli.\newline 
\indent \textit{Ścieżka} w grafie to dowolny ciąg wierzchołków, taki że 
$$ x_{1} \sim x_{2} \sim \dots \sim x_{n}$$ \newline 
\indent \textit{Spójny graf} - graf, w którym każdą parę wierzchołków łączy pewna ścieżka. \textit{Graf skończony} ma skończoną ilość wierzchołków ($\# V < \infty$). W tym wprowadzeniu rozważamy wyłącznie skończone spójne grafy. \newline 
\indent\textit{Graf pełny} - graf w którym if $x \sim y$ dla każdego $ x,y \in V$. Inaczej graf nazywa się \textit{niepełnym}.\newline 
\indent Graf nazywa się \textit{dwudzielnym}, jeżeli istnieje podział jego wierzchołków $V = V_{1} \cup V_{2}$ taki że z $x \sim y$  $(x,y \in V)$ wynika albo $ x \in V_{1}$, $y \in V_{2}$, albo $x \in V_{2}$, $y \in V_{1}$.\newline 
\indent Będziemy też rozpatrywać wagi \textit{znormalizowane}: 
\begin{eqnarray} 
p(x,y) = \frac{b(x,y)}{b(x)},
\end{eqnarray} 
gdzie $b(x) = \sum_{y} b(x,y)$. Trzeba uważać, że ogólnie $p(x,y) \neq p(y,x)$. 
\section{PODSTAWOWE OPERATORY RÓŻNICZKOWE} 
\indent Rozważmy następujący zbiór funckji na wierzchołkach grafu ważonego
$$\Im = \{f|f: V \rightarrow \mathbb{R}\}$$\newline 
\indent  Naszym głównym zainteresowaniem będzie znormalizowany operator Laplace'a (laplasjan) oraz operator prawdopodobieństwa $\Im$. \textit{Operator Laplace'a (laplasjan)} jest zdefiniowany jako 
\begin{eqnarray} 
\mathcal{L}f(x) = \sum_{y \in V} (f(x) - f(y)) \frac{b(x,y)}{b(x)}= \sum_{y \in V} (f(x) - f(y))p(x,y) 
\end{eqnarray} 
dla wszystkich $f \in \Im$.\newline 
\indent Operator $\mathcal{P= I - L}$ dla $f \in \Im$ nazywa się \textit{operatorem prawdopodobieństwa}. \newline
\newline
\textbf{Definition 1.} Operator \textit{różnicy} jest zdefiniowany jako \newline
$$\nabla f = f(y) - f(x)$$ \newline
Operator różnicy jest dyskretnym analogiem pochodnej. Z(3) wynika, że \newline
Opróćz tego, wprowadzimy nastepujące oznaczenie: \newline
\indent Jeżeli , to  jest zbiorem pustym, wiec ostatni wyraz w (5) znika i otrzymujemy \newline
\textbf{Theorem 2} (Formula Greena). \textit {Dla każdych dwóch fukcnji f,g:...  i dla każdego...  spełniają sie następujące tożsamości:} \newline
gdzie w ostatnim wierszu zmieniliśmy notację zmiennich x i y w pierwszej sumie. Dodając do siebie ostatnie dwa wiersze i dzieląc przez 2, otrzymujemy (5).\newline
\textbf{ Corollary 3.} \textit{Dla każdej funkcji f:}\newline
\section{PROBLEM DIRICHLETA}
\indent nazywa sie \textit{problemem Dirichleta}. Ten problem jest dyskretną wersją ciągłego problemu Dirichleta. \newline
\newline
\textbf{Theorem 4} \textit{Problem Dirichleta }(8)\textit{ zawsze ma dokładnie jedno rozwiazanie v:} \newline
\indent Punktem kluczowym dowodu twierdzenia 4 jest następna lemma.\newline
\textbf{Lemma 5} (Zasada maksimuma i minimuma) \textit {Niech} $\sum \int$
\end{document} 