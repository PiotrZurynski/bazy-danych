\documentclass[10pt,a4paper]{article}
\usepackage[utf8]{inputenc}
\usepackage[MeX]{polski}
\title{pracadowa}
\author {IMIE NAZWISKO}
\begin{document}
% Wstawienie autora i~tytułu do składu:
\maketitle
\noindent
\raggedright
\textsc {\scriptsize 1. PRZYKLAD TEKSTU PO ANGIELSKU: A\texttt{ESOP} "T\texttt{HE} H\texttt{ARE} \texttt{AND THE} T\texttt{ORTOISE".}}
\\ [0.2cm]
\parindent 0.5 cm
\small One day the Hare laughed at the short feet and slow speed of the Tortoise.The Tortoise replied:
\\
\textit {"You may be as fast as the wind, but I will beat you in race"}
\\

\small The Hare thought this idea was impossible and he agreed to the proposal. It 
was agreed that the Fox should choose the course and decide the end.

\small The day for the race came, and the Tortoise and Hare started \textbf{together}.

\small \textbf{The Tortoise never stopped for a moment}, walking slowly but steadily, right to the end of the course. The Hare ran fast and stopped to lie down for a rest. But he fell fast asleep. Eventually, he woke up and ran as fast as he could. But when he reached the end, he saw the Tortoise there already, sleeping comfortably after her effort.
\\
\begin{center}
\textsc { \normalsize  2.Rozmiary czcionki}
\end{center}

\small My przeczytaliśmy wszystkie polecenia na stronie:
\\
\begin{flushleft}
\small {$https://www.overleaf.com/learn/latex/Font\_sizes\%2C\_families\%2C\_and\_styles.$}
\end{flushleft}

\small Teraz mozemy to wykorzystac.
\\
\textsc {\tiny Czcionka jest tutaj bardzo,mała trzeba powiekszyc ten tekst.
\end{document} 


\author{Rozmiary Czcionki}

\textrm{ \emph{„You may be as fast as the wind, but I will beat you in race"}}